\documentclass{resume}

\usepackage{textcomp}

\renewcommand{\categoryfont}{\sc}

%
% set the space used for category titles here:
% use the same value for oddsidemargin and marginparwidth [the latter 
%               will be reset to account for marginparsep]
% 
\setlength{\oddsidemargin}{1in}
\setlength{\marginparwidth}{1in}
% 
% calculate other dimensions [textwidth and evensidemargin] 
% in function of oddsidemargin and marginparwidth: 
% would be nicer to put in the class file...
%
\addtolength{\marginparwidth}{-\marginparsep}
 \setlength{\evensidemargin}{\oddsidemargin}
 \setlength{\textwidth}{\paperwidth}
 \addtolength{\textwidth}{-2in}
 \addtolength{\textwidth}{-2\oddsidemargin}
 \addtolength{\textwidth}{\marginparwidth}
 \addtolength{\textwidth}{\marginparsep}
%
%
\setlength{\topmargin}{-0.5in}
%\setlength{\textheight}{9.5in}
%
%
\renewcommand{\labelcitem}{$\diamond$}
\renewcommand{\labelitemi}{$\cdot$}
\newcommand{\first}{$1^{\mbox{\scriptsize st}}$\ }
\newcommand{\second}{$2^{\mbox{\scriptsize nd}}$\ }
\newcommand{\third}{$3^{\mbox{\scriptsize rd}}$\ }

\author{Pieter Vermeesch}

% ------ Address --------------------------------------------------------

\address{Department of Earth and Planetary Sciences \\
Birkbeck College, University of London\\
\mbox{\small\tt p.vermeesch@ucl.ac.uk}\\
        \mbox{\small\tt http://ucl.ac.uk/$\sim$ucfbpve}}{
Malet Street\\
London WC1E 7HX\\
tel: +44 (0)20 7679 2418\\
fax: +44 (0)20 7679 2867
}

\begin{document}
\maketitle

% ------- Education ---------------------------------------------------

\begin{category}{Education}
\citembullet Stanford University, Stanford, CA, United States.\\
Ph.D. in Geology, June 2005.\\
Thesis title: {\em Contributions to detrital thermochronology.}
\citembullet Massachusetts Institute of Technology, Cambridge, MA, United States. \\
M.Sc. in Geosystems, July 2000.\\
Thesis title: {\em Thermal evolution of a compositionally stratified
Earth,  including plates.}
\citembullet Universiteit Gent, Gent, Belgium.\\
B.Sc. in Geology, July 1999.\\
Thesis title: {\em Studie van  de recente evolutie van het Issyk Kul
Bekken  (Tien  Shan, Kirghistan)  met  behulp van  warmtefluxmetingen,
hoge-resolutie         reflectie-seismische        profielen        en
veldwaarnemingen.}
\end{category}

% -------- Work experience --------------------------------------------

\begin{category}{Work \\experience}
\citembullet RCUK Academic Fellow, Birkbeck College (2007 -- present)\\
Teaching: Introduction to Geology, Assessed field techniques I
\citembullet Marie-Curie postdoctoral researcher, ETH-Z\"{u}rich (2005 -- 2007)
\citembullet Research Assistant, Stanford University (2001 -- 2004)\\
Assisted students and faculty in the School of Earth Sciences GIS lab.
\citembullet Field Engineer, Schlumberger Wireline, Ras Shukeir, Egypt (2000 -- 2001)\\
Performed borehole measurements on- and off-shore.
\end{category}

% --------- Research ----------------------------------------------------

\begin{category}{Research interests}
\citemnobullet (Detrital) geochronology and thermochronology, terrestrial cosmogenic nuclides, 
statistical analysis of geochemical data, remote sensing, aeolian geomorphology.
\end{category}

% ------- Honors and Awards ---------------------------------------------------

\begin{category}{Honors and awards}
\citembullet 2011 Birkbeck College's Ronald Tress Prize for excellence in research 
\citembullet 2009 Geology Exceptional Reviewer
\citembullet Most cited paper award 2004-2007 for the EPSL article 
``How many grains are needed for a provenance study?''
\citembullet Best oral presentation at the annual Stanford School of Earth Sciences research review
\citembullet Francqui Fellow of the Belgian American Educational Foundation 
%\citembullet Fullbright Fellowship (declined)
\citembullet Vall\`{e}re Billet Award for best geology student of Ghent University
\end{category}

% ------- Skills ------------------------------------------------------

%\begin{category}{Skills}
%\citembullet Languages: Dutch, English
%\citembullet Equipment: Electron Microprobe,  SEM, geochronology by $^{40}$Ar/$^$39}Ar, 
%cosmogenic $^3$He \& $^{21}$Ne, U-Th-Pb (SHRIMP-RG and LA-ICP-MS), fission track and (U-Th)/He methods.
%\citembullet Computing: C/C++, VBA-Excel, MATLAB, R/S-Plus, Java/Javascript,
%ArcView/ArcGIS, \LaTeX.
%\end{category}

% -------- Community Service --------------------------------------------

\begin{category}{Community service}
  \citembullet Reviewer for Tectonics, Geology, GSA Bulletin, Chemical Geology, Terra Nova, 
Remote Sensing of Environment, Quaternary Geochronology, Earth and Planetary Science Letters, 
Journal of Sedimentary Research, Mathematical Geosciences, Geomorphology, Journal of Geophysical Research,
Journal of Quaternary Science, Journal of the Geological Society
  \citembullet Organised the CRONUS-EU field trip on applications of cosmogenic nuclides to the 
  glacial geomorphology of the Alps.
  \citembullet Organising committee member of the 69th annual Meteoritical Society meeting in Z\"{u}rich.
  \citembullet Convener of 2006 AGU fall meeting session T27: ``Advances in (U-Th)/He geochronology''.
\end{category}

\clearpage

% -------- Research Grants --------------------------------------------

\begin{category}{Research Grants}
  \citembullet 2011-2014: NERC Standard Grant for ``A novel approach to in-situ zircon U-Th-He geochronology'' (\pounds 528,527, in review) 
  \citembullet 2011-1014: NERC Standard Grant for ``Dust storms and Chinese loess over the last 22 million
years.'' (\pounds 360,996)
  \citembullet 2011-2016: ERC Starting Grant for ``K-Ar and Ar-Ar geochronology by stepwise dissolution.''
(\texteuro 580,992)
  \citembullet 2011: NERC-CIAF award for ``Sediment storage and recycling in the Namib Sand Sea and
its Miocene ancestor'' (\pounds 13,200)
  \citembullet 2009: NERC-CIAF award for 
``Measuring the residence time of sand in Namib dunes with cosmogenic nuclides.''
(\pounds 8,565)
\end{category}


% -------- Publication --------------------------------------------

\begin{category}{Peer reviewed articles}

\citemnobullet Vermeesch, P., 2011. A 45 year time series of Saharan dune mobility from
remote sensing (in preparation for submission to Science).

\citemnobullet Vermeesch, P. and Carter, A., 2011. A simplified method for in-situ U-Th-He dating.
Geochimica et Cosmochimica Acta (in review).

\citemnobullet Vermeesch, P., 2011. Reply to five comments on ``Lies, damned lies,
and statistics (in geology), Eos (in press).

\citemnobullet Vermeesch, P., Fenton, C.R., Kober, F., Wiggs, G.F.S.,
    Bristow, C.S. and Xu, S., 2010. One million year residence time of Namib
    dune sand from cosmogenic nuclides, Nature Geoscience, 3, 862-865.

\citemnobullet Vermeesch, P., 2010. HelioPlot, and the treatment of
overdispersed (U-Th-Sm)/He data, Chemical Geology, 271 (3-4) 108-111

\citemnobullet Vermeesch, P., 2009. Lies, damned lies, and statistics
(in Geology), Eos, 90 (47), p.443

\citemnobullet Vermeesch, P., Baur, H., Heber, V. S., Kober, F.,
  Oberholzer, P., Sch\"{a}fer, J. M., Schl\"{u}chter, C., Strasky, S.,
  and Wieler, R., 2009. Cosmogenic $^3$He and $^{21}$Ne measured in
  quartz targets after one year of exposure in the Swiss Alps, Earth
  and Planetary Science Letters, 284, 3-4, 417-425.

\citemnobullet Vermeesch, P., 2009, RadialPlotter: a Java application
  for fission track, luminescence and other radial plots, Radiation
  Measurements, 44, 4, 409-410.

\citemnobullet Vermeesch, P., Avigad, D. and McWilliams, M., 2009. 500
Myr of thermal history elucidated by multi-method detrital
thermochronology of North Gondwana Cambrian sandstone (Eilat area,
Israel), Geological Society of America Bulletin, v. 121, n. 7/8,
1204-1216.

\citemnobullet Vermeesch, P. and Drake, N., 2008. Remotely sensed dune
celerity and sand flux measurements of the world's fastest barchans
(Bod\'{e}l\'{e}, Chad): Geophysical Research Letters, 35, L24404.

\citemnobullet Vermeesch, P., 2007, Quantitative geomorphology of the
White Mountains (California), using detrital apatite fission track
thermochronology: Journal of Geophysical Research -- Earth Surface,
112, F03004.

\citemnobullet Vermeesch, P., 2007, CosmoCalc: an Excel add-in for
cosmogenic nuclide calculations: Geochemistry, Geophysics, and
Geosystems, Vol. 8, Q08003.

\citemnobullet Vermeesch, P., 2007, Reply to Comment by Agrawal and
Verma on .Tectonic classification of basalts with classification
trees., Geochimica et Cosmochimica Acta, v.71, p.3391-3392

\citemnobullet Vermeesch, P., Seward, D., Latkoczy, C., Wipf, M.,
G\"{u}nther, D. and Baur, H., 2007, $\alpha$-emitting mineral
inclusions in apatite, their effect on (U-Th)/He ages, and how to
reduce it: Geochimica et Cosmochimica Acta, v. 71, no. 7,
pp. 1737-1746.

\citemnobullet Vermeesch, P., 2006, Tectonic discrimination diagrams
revisited: Geochemistry, Geophysics, and Geosystems, 7, Q06017.

\citemnobullet Vermeesch, P., Miller, D. D., Graham, S. A., De Grave,
J., and McWilliams, M. O., 2006, Multi-method detrital
thermochronology of the Great Valley Group near New Idria (California)
: Geological Society of America Bulletin, v. 118, no. 1, pp. 210-218

\citemnobullet Vermeesch, P., 2006, Tectonic discrimination of basalts
with classification trees: Geochimica et Cosmochimica Acta, v. 70,
no. 7, pp. 1839-1848.

\citemnobullet Vermeesch, P., 2005, Statistical uncertainty associated with histograms
in the Earth Sciences, Journal of Geophysical Research -- Solid Earth, Vol 110, B02211.

\citemnobullet Vermeesch,  P., 2004,  How many  grains  are needed  for a  provenance
study? Earth and Planetary Science Letters, v.224, n.3-4, pp. 441-451.

\citemnobullet Vermeesch, P., Poort, J., Duchkov,  A. D., Klerckx, J., and De Batist,
M., 2004,  Lake Issyk-Kul (Tien  Shan): Unusually low heat-flow  in an
active intermontane basin.: Russian Geology and Geophysics, v.45, n.5,
pp. 616-625.

\citemnobullet Vermeesch, P., 2003,  A second look at the geologic  map of China: the
``Sloss approach''. International Geology Review, v.45, p. 119-132.
\end{category}

\end{document}
