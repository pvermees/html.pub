\documentclass{article}
\begin{document}

\section*{Summary}

The Wasserstein distance is shown to be an appropriate dissimilarity metric for comparing distributional data such as detrital mineral ages. Using synthetic and real data we compare the Wasserstein distance to the commonly used Kolmogorov-Smirnov distance. The results are, in general, similar, but where they differ the Wasserstein distance is found to have more geologically sensible results. Code required to calculate the Wasserstein distance between distributions is provided in python and R.

\end{document}