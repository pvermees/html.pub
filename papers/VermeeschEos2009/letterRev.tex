\documentclass[a4]{letter}
\usepackage{graphicx}
\pagestyle{empty}
\usepackage{fullpage}

\pagestyle{empty}
\signature{Pieter Vermeesch}
\date{18 June 2009}
\address{
Pieter Vermeesch\\
Research School of Earth Sciences\\
Birkbeck and University College London\\
Malet Street, London, WC1E 7HX, UK\\
email: p.vermeesch@ucl.ac.uk
}

\begin{document}
\begin{letter}{Barbara T. Richman \\
Executive Editor, {\it Eos}}

\opening{Dear Ms. Richman,} 

I was happy to learn that you found my manuscript ``Hypothesis testing
in geology'' suitable for publication in {\it Eos}, albeit pending
modification. I was particularly pleased with the approval by Albert
Tarantola, whose opinion I respect a lot.  His comment that ``there
are three kinds of lies'' prompted me to change the title of my
article to ``lies, damned lies, and statistics (in geology)'',
following Mark Twain's famous phrase.  I hope that this title will
catch the attention of the casual Eos reader and removes the need for
the provocative and somewhat aggressive opening paragraph of the
original manuscript, which reviewer 3 did not like.  I believe I have
appropriately addressed reviewer 3's other concerns as well.  First of
all, I followed his suggestion and moved the sentence ``statistically
significant is not the same as geologically significant'' to the
opening paragraph. I reduced the ``wanderment into philosophical
realms'' by shortening the second and completely removing the fourth
paragraph of the original manuscript. To make the paper more
accessible to those with no prior knowledge of statistics, I expanded
the explanation of the chi-square test, giving explicit examples of a
null hypothesis (``average global temperature has stayed constant
since 1900'') and two alternative hypotheses in the second paragraph.
I have also introduced the concept of one-sided and two-sided
hypothesis tests. I now believe that paragraphs two and three of my
manuscript comprise one of the shortest introductions to statistical
hypothesis testing ever published. To stress the importance of sample
size in determining the outcome of statistical hypothesis tests, the
revised manuscript considers the consequences of reducing the
earthquake database by a factor of 10, which results in a failure to
reject the null hypothesis. I think that this short addition to the
last paragraph puts the nail in the coffin of hypothesis testing in
terms that are understandable even to the less numerate readers.

I hope that you will find the ammendments to your satisfaction and
look forward to hearing back from you,

\closing{Sincerely yours,}

~~~~~~~~~~~~~~~~~~~~~~~~~~~~~~~~~~~~~~~~~~~~~   
\includegraphics[width=0.5\textwidth]{F:/signature.pdf}

\end{letter}
\end{document}
