\documentclass{article}
\title{Cosmogenic $^3$He and $^{21}$Ne measured in quartz
targets\\ after one year of exposure in the Swiss Alps}
\author{Pieter Vermeesch$^1$
%\footnote{Corresponding author; current affiliation:
%Birkbeck, University of London; p.vermeesch@ucl.ac.uk; +44 (0)20 7679 3406}, 
%Heinrich Baur$^1$, Veronika S. Heber$^1$,Florian Kober$^1$,Peter Oberholzer$^1$\\
%Joerg M. Schaefer$^1$,Christian Schl\"{u}chter$^2$, Stefan Strasky$^1$, Rainer Wieler$^1$\\~\\
%{\small$^1$Institute of Isotope Geology and Mineral Resources, ETH-Zurich, Switzerland}\\
%{\small$^2$Institute of Geological Sciences, University of Bern, Switzerland}}
\footnote{Institute of Isotope Geology and Mineral Resources, ETH-Zurich, Switzerland}
\footnote{Corresponding author; current affiliation:
Birkbeck, University of London; p.vermeesch@ucl.ac.uk; +44 (0)20 7679 3406}, 
Heinrich Baur\footnotemark[1],
Veronika S. Heber\footnotemark[1]
\footnote{Current affiliation: Earth and Space Sciences, UCLA, Los Angeles, United States}\\
Florian Kober\footnotemark[1],
Peter Oberholzer\footnotemark[1]
\footnote{Current affiliation: Baugeologie und Geo-Bau-Labor Chur, Switzerland}, 
Joerg M. Schaefer\footnotemark[1]
\footnote{Current affiliation: Lamont Doherty Earth Observatory, USA},\\
Christian Schl\"{u}chter\footnote{
Institute of Geological Sciences, University of Bern, Switzerland}, 
Stefan Strasky\footnotemark[1]
\footnote{Current affiliation: Swiss Geological Survey, Switzerland},
Rainer Wieler\footnotemark[1]\\
}
\usepackage{fullpage,natbib,graphicx,lineno,setspace,wasysym}
\addtolength{\tabcolsep}{-3pt}
\pagestyle{empty}
\begin{document}
\onehalfspace
\linenumbers
\maketitle
\begin{abstract}
  All currently used scaling models for Terrestrial Cosmogenic Nuclide
  (TCN)  production  rates  are  based  on  neutron  monitor  surveys.
  Therefore,  an  assumption  underlying   all  TCN  studies  is  that
  production rates  are directly proportional to  secondary cosmic ray
  intensities  for  all cosmogenic  nuclides.   To  test this  crucial
  assumption,   we  measured  cosmogenic   $^3$He  and   $^{21}$Ne  in
  artificial  quartz targets after  one year  of exposure  at mountain
  altitudes in the  Swiss Alps.  The targets were  inconel steel tubes
  containing  one  kg  of  artificial quartz  sand  (250-500  $\mu$m),
  degassed  for  one  week  at  700 $^{\circ}$C  in  vacuum  prior  to
  exposure.  From August 2006 until  August 2007, ten of these targets
  were exposed at five  locations in Switzerland and Italy: Z\"{u}rich
  (556m), Davos (1560m), S\"{a}ntis (2502m), Jungfraujoch (3571m), and
  Monte Rosa (4554m).  Additionally, a  sixth set of two blank targets
  was  kept  in  storage  and  effectively shielded  from  cosmic  ray
  exposure.  Cosmogenic noble gases  were measured at room temperature
  and  at 700 $^{\circ}$C.   Up to  9\% of  the cosmogenic  $^3$He was
  measured in  the cold step,  indicating that $^3$He diffuses  out of
  quartz  at room  temperature on  short time  scales.   The remaining
  $^3$He and all  $^{21}$Ne was released at 700  $^{\circ}$C, as shown
  by  a repeat  measurement  at  800 $^{\circ}$C  for  the Monte  Rosa
  target, which yielded no  additional cosmogenic helium and neon.  As
  expected,  the Monte  Rosa target  contained the  highest cosmogenic
  nuclide  content, with  1.56  $\pm$ 0.07  $\times$  10$^6$ atoms  of
  excess  $^3$He and  4.5 $\pm$  1.2 $\times$  10$^5$ atoms  of excess
  $^{21}$Ne  (all errors  are 2$\sigma$).   The raw  measurements were
  corrected  for non-atmospheric blanks,  shielding (roof  + container
  wall), tritiogenic  helium and  solar modulation (normalised  to the
  average  neutron  flux  over  the  past  five  solar  cycles).   The
  $^{3}$He/$^{21}$Ne production rate ratio  of 7.0 $\pm$ 0.9 indicates
  that  cosmogenic  $^{3}$He  production  by the  container  walls  is
  negligble.  The main goal of the artificial target experiment was to
  determine the  production rate attenuation length.   Because all our
  targets  had an identical  design and  were exposed  under identical
  conditions, all  systematic errors cancel out in  the calculation of
  an  attenuation  length.  Our  best  estimates  for  the $^3$He  and
  $^{21}$Ne attenuation  lengths are 132.7 $\pm$ 5.7  g/cm$^2$ and 135
  $\pm$ 25  g/cm$^2$, respectively, agreeing very  well with currently
  used  scaling models.   We conclude  that TCN  production  rates are
  indeed proportional to neutron  monitor count rates, and that $^3$He
  and $^{21}$Ne  production rates follow the  same altitudinal scaling
  relationships  as   the  cosmogenic  radionuclides.    Finally,  the
  measurements were  scaled to sea  level and high latitude  using the
  empirical  attenuation  length,  yielding weighted  mean  production
  rates  of 110.5  $\pm$ 6.7  at/g/yr for  $^3$He and  15.4  $\pm$ 2.1
  at/g/yr  for  $^{21}$Ne.    Despite  the  significant  uncertainties
  associated with the corrections  for shielding, solar modulation and
  especially the $^3$He/$^3$H branching  ratio, these estimates are in
  good agreement with production rates derived from long-term exposure
  experiments   at  natural   calibration   sites  and   physics-based
  simulations.
\end{abstract}

\begin{center}
{\it Keywords: cosmogenic nuclides, attenuation length, neon, helium, quartz}
\end{center}

\section{Introduction}\label{sec:intro}

All currently  used scaling models for  Terrestrial Cosmogenic Nuclide
(TCN)  production   rates  are   based  on  neutron   monitor  surveys
\citep{lal1991,   stone2000,   dunai2000,   dunai2001,   desilets2003,
  pigati2004,  lifton2008}.  Therefore,  an assumption  underlying all
cosmogenic  nuclide  studies is  that  production  rates are  directly
proportional to  secondary cosmic  ray intensities for  all cosmogenic
nuclides.   Several   efforts  are  underway  to   test  this  crucial
assumption by TCN production rate calibrations in the framework of the
CRONUS-EU and CRONUS-Earth initiatives.  The bulk of this work is done
on  landforms  of known  age  \citep{desilets2006b}.  These  so-called
`natural  calibration  targets'  are  the  method of  choice  for  the
calculation   of  accurate  TCN   production  rates   integrated  over
millennial  time  scales but  are,  unfortunately,  often affected  by
poorly  constrainted factors  such as  shielding and  erosion.   It is
notoriously  hard to  find vertical  transects of  natural calibration
sites  that  allow  the  calculation of  production  rate  attenuation
lengths.   Herein  lies   the  complementary  strength  of  artificial
calibration targets. Because the exposure conditions of the latter are
either  known or  constant, all  systematic errors  cancel out  in the
calculation of a production  rate attenuation length.  We here present
the first  results of an  artificial target experiment  measuring, for
the first  time, cosmogenic $^3$He  and $^{21}$Ne in quartz  after one
year of  exposure at mountain  altitudes in the Swiss  Alps.  Previous
artificial   target   experiments  have   mainly   focused  on   water
\citep{lal1960,  nishiizumi1996, graham2000, brown2000},  although one
pilot  experiment used  a  silicate glass  \citep{graf1996}.  We  used
quartz as  the target material, because  it is the  most commonly used
mineral for  exposure dating and  both cosmogenic helium and  neon are
produced and retained in the target container.
\\

Our  project has  a history  of more  than ten  years. A  first target
design  was  developed  back  in  1997.  These  were  stainless  steel
containers with  14 cm radius  and 45 cm  height, filled with 4  kg of
industrial  quartz   sand  of   natural  origin  (Fluka,   no.   83340
\citep{schaefer2000, kober2004}).   The targets were  heated to $>$800
$^{\circ}$C  under vacuum  for  a  week in  order  to ensure  complete
degassing prior to exposure, and double sealed with a valve and copper
tube clamp to prevent  atmospheric leaks during exposure.  Fourteen of
these targets  were exposed  at seven different  locations for  two to
four years.  Two of them were measured, one unexposed blank target and
one target  that had been exposed  at Jungfraujoch, at  an altitude of
3571 m.   The pilot experiment was  aborted after the  neon and helium
compositions  of these  two  targets were  found  to be  a mixture  of
cosmogenic  and  other  components  \citep{kober2004}. There  are  two
reasons  why  the first  target  design  failed.   The presence  of  a
`trapped'  component  (neon plotting  above  the  mixing line  between
atmospheric  and  cosmogenic  components,  the  so-called  `spallation
line';  Niedermann, 2002)  indicates that  pre-exposure  degassing was
insufficient, and  that the quartz  did not reach the  600 $^{\circ}$C
degassing temperature of neon \citep{niedermann2002}.  The presence of
a `nucleogenic'  component (neon  plotting below the  spallation line)
indicates that  despite the purity  of the industrial quartz  sand, it
still contained sufficient alpha producing  U and Th (120 and 172 ppb,
respectively) to  compromise the helium and  neon measurements.  These
observations  led to  the development  of a  second  generation target
design.  \\

The  effectiveness of  the revised  target  design was  verified in  a
custom-built prototype  container (Section \ref{sec:design}).   Ten of
these targets were exposed at  different elevations in the Swiss Alps,
at    altitudes     ranging    from    556     to    4554m    (Section
\ref{sec:installation}). Cosmogenic $^3$He and $^{21}$Ne were measured
after one year of exposure, using a custom-built mass spectrometer and
an  optimised  measurement  routine (Section  \ref{sec:measurements}).
Data  reduction  included   corrections  for  non-atmospheric  blanks,
shielding,   solar  modulation,   and   tritiogenic  helium   (Section
\ref{sec:datareduction}).  $^3$He  and $^{21}$Ne were  measured in two
steps at room  temperature and at 700 $^{\circ}$C.  Most of the $^3$He
and  all of  the  $^{21}$Ne were  measured  in the  hot step  (Section
\ref{sec:results}).   The altitude  dependency of  the  TCN production
rates  was  quantified by  plotting  them  against atmospheric  depth,
yielding  attenuation  lengths  that  are in  perfect  agreement  with
existing  scaling models (Section  \ref{sec:attenuation}).  Production
rates were scaled  to sea level and high latitude  and agree well with
previous   determinations  on   natural  calibration   sites  (Section
\ref{sec:P}).  We conclude  this paper with an outlook  to the future,
when  duplicate  artificial targets  will  be  used  to determine  the
$^3$He/$^3$H branching ratio and  we will monitor cosmogenic noble gas
production    rates   over    an   entire    solar    cycle   (Section
\ref{sec:conclusions}).

\section{Methods}\label{sec:methods}

\subsection{Target design}\label{sec:design}

The first  generation targets suffered from  sub-optimal degassing and
impure quartz (Section \ref{sec:intro}).   Both of these problems were
addressed in  the second generation  target design.  To  eliminate the
trapped neon component and ensure optimal degassing, the radius of the
seemless stainless steel (grade  1.4301) canisters was reduced from 14
to 6  cm (Figure  \ref{fig:design}.c), and in  order to  eliminate the
nucleogenic component,  we used artificially grown  quartz crystals of
optimal purity  (supplied by Morion Company, USA),  which were crushed
to 250-500  $\mu$m sand  size (Figure \ref{fig:design}.a),  and rinsed
with water and acetone.   Gamma ray spectrometry measurements revealed
U and  Th concentrations $<$16  and $<$49 ppb, respectively,  which is
below  the detection limit  of the  method and  also below  the levels
measured in the Fluka  quartz sand \citep{strasky2008}.  To verify the
effectiveness  of  the  new  target  design,  two  thermocouples  were
installed  in a  prototype container  filled  with 800  g quartz  sand
(Figure  \ref{fig:design}.c).   After a  heating  period of  $\sim$2.5
hours at  an external temperature of 900  $^{\circ}$C, the temperature
reached  by the  quartz in  the innermost  part of  the  container was
$\sim$850 $^{\circ}$C, well above the degassing temperatures of helium
and   neon  \citep[$<$  600   $^{\circ}$C,][]{niedermann2002}  (Figure
\ref{fig:design}.d).   To reduce the  blank, the  external temperature
for  the   actual  target  measurements  was  later   reduced  to  700
$^{\circ}$C,   which  should   yield   $\sim$650  $^{\circ}$C   quartz
temperatures.   As was  the  case for  the  first generation  (Section
\ref{sec:intro}),  also  the  second  generation targets  were  double
sealed  by a `bellows-sealed'  Swagelok$^{\tiny\textregistered}$ valve
connected to a copper tube with a stainless steel pinch-off clamp.

\subsection{Pre-treatment and installation}\label{sec:installation}

One kg of  the artificial quartz sand was  degassed inside the targets
for one  week at 700 $^{\circ}$C  in vacuum prior to  exposure using a
custom-built  furnace (Figure  \ref{fig:design}.b).  The  targets were
rolled  in  bubble wrap  and  placed  in  fibreglass cable  trays  for
protection  against the weather,  and were  installed in  a horizontal
position to  minimise self-shielding.  In August of  2006, two targets
were  exposed at  each  of five  locations:  Z\"{u}rich (556m),  Davos
(1560m),  S\"{a}ntis  (2502m), Jungfraujoch  (3571m),  and Monte  Rosa
(4554m).   All  of  these   locations  (except  for  Monte  Rosa)  are
meteorological   observatories  of   the  Swiss   Federal   Office  of
Meteorology  and Climatology (MeteoSwiss),  which were  kept snow-free
during the winter of 2006-2007.  The Z\"{u}rich, Davos, and S\"{a}ntis
targets  were  installed  outside  and  secured  to  the  railings  of
meteorological equipment.   Because of the extremely  high wind speeds
at  the Jungfraujoch  and Monte  Rosa sites,  those targets  were kept
inside.  Additionally, a sixth set  of two blank targets was stored in
the  basement  of a  10-storey  building  housing  the ETH  noble  gas
laboratory,  $\sim$15m below  street level,  and  effectively shielded
from cosmic  ray exposure.  Exactly  one year later, the  targets were
retrieved and subsequently measured.

\subsection{Measurements}\label{sec:measurements}

Even in one kilogram of quartz and at mountain altitudes, the expected
amounts of cosmogenic  gas are extremely low, on the  order of tens of
thousands of  atoms in a volume  of more than four  litres. To measure
such minute  amounts of  noble gases,  we used a  unique kind  of mass
spectrometer  developed at  ETH-Z\"{u}rich, which  is equipped  with a
compressor  source  \citep{baur1999}.  The  compressor  consists of  a
magnetically levitated  rotor, spinning at  1500 Hz, which  forces the
gas along spiral grooves in the inner wall of the stator.  The neutral
gas then enters the ionization volume and gets accelerated towards the
magnet  and ion  detectors.  The  compressor  source acts  as a  pump,
consuming a much larger portion  of the sample gas than a conventional
mass spectrometer, and resulting in  a two orders of magnitude gain in
sensitivity.  \\

Cosmogenic noble gases were measured  in two steps at room temperature
and  700   $^{\circ}$C.   Before  commencing   the  measurements,  the
connection of the  artificial targets to the gas  preparation line via
the copper  tube and a flexible  bellow was baked and  degassed for 24
hours.   Pressure in  the extraction  line and  mass  spectrometer was
maintained at  ultrahigh vacuum conditions of  $\sim$ 10$^{-10}$ mbar.
To protect the extraction line and mass spectrometer against potential
atmospheric leaks, a `dummy run' was made by measuring the gas between
the stainless  steel pinch-off and  the valve.  Because the  amount of
this gas was always very small ($\sim$ 20,000 atoms $^3$He) and had an
atmospheric  composition   ($^3$He/$^4$He  $\approx$  2.1   $\pm$  0.9
$\times$ 10$^{-6}$), it is not reported in the data tables.  Next, the
valve was opened and the target volume equilibrated with a cold finger
filled  with activated  charcoal and  cooled  by liquid  N$_2$ for  15
minutes before releasing the gas into the gas purification line, where
it was  cleaned by one Ag  and three Zr/Ti getters  and two additional
cold fingers for 30 minutes.  Neon and helium were measured separately
by freezing  the neon to  a cryogenic trap  at 12.7 K for  15 minutes.
The gas  was expanded  into the mass  spectrometer while exposed  to a
final cold finger and getter.   To monitor the sensitivity of the mass
spectrometer (which changed $<$  2 \%/day), calibrations (`fast cals')
were done  every morning and evening,  using a fixed gas  amount of an
internal lab standard.  \\

The  sample  gas  was  pumped   into  the  ionization  volume  by  the
aforementioned  magnetically   levitated  compressor,  ionized   by  a
Baur-Signer source \citep{baur1980} and accelerated through a magnetic
sector mass-spectometer  with a trajectory  radius of 210 mm  and mass
resolution set to 600 in order to resolve $^3$He from the HD molecule.
The $^3$He signal was digitally measured on an ion counter in 20 steps
of 60 seconds while $^4$He was  measured in analogue mode on a Faraday
cup  in 20 steps  of 40  seconds.  These  long measurement  times were
necessary  to   overcome  the  relatively   poor  counting  statistics
associated with our extremely  low signals.  Extrapolation to the time
of gas inlet was achieved by linear regression of the data.  Next, the
remaining helium was pumped away  and the cryogenic trap heated to 50K
in  order to  release  the  neon.  Whereas  the  helium isotopes  were
simultaneously  analysed,  $^{20}$Ne,  $^{22}$Ne  and  $^{21}$Ne  were
measured in peak  jumping mode on the ion counter in  15 cycles of 20,
30 and  60 seconds, respectively.   H$_2$O, $^{40}$Ar and  CO$_2$ were
monitored  and  interference  corrections  were  made  for  masses  20
(2$\permil$ of H$_2$O  and 0.45 $\permil$ of $^{40}$Ar)  and 22 (0.044
$\permil$ of CO$_2$).   $^{20}$Ne was extrapolated to the  time of gas
inlet  by   exponential  regression.   The   $^{21}$Ne  and  $^{22}$Ne
abundances were  calculated by monitoring  the $^{22}$Ne/$^{20}$Ne and
$^{21}$Ne/$^{20}$Ne ratios  over the  course of each  analysis.  These
so-called  `local ratios'  were  constant for  the  first five  target
measurements, which  took place in November 2007.   The remaining four
targets were  measured several months  later as a result  of technical
problems unrelated  to our experiment.  The  mass spectrometer behaved
slightly  differently  when  the  measurements resumed,  resulting  in
$^{22}$Ne/$^{20}$Ne   and   $^{21}$Ne/$^{20}$Ne   local  ratios   that
increased with time.   To account for this change  in behaviour, a new
blank  target  (T16)  was  measured  after  completion  of  the  final
measurements,  so as  to  ensure internal  consistency  of the  target
measurements  and  the  blank-target  based  fractionation  correction
(Section \ref{sec:blank}).  \\

After completion of the neon  measurement, the cryogenic trap and mass
spectrometer  were isolated  from the  extraction line,  the remaining
`cold' target  gas was pumped away,  and the furnace was  switched on. 
An  external  temperature  of  700 $^{\circ}$C  (corresponding  to  an
internal     temperature    of    $\sim$650     $^{\circ}$C,    Figure
\ref{fig:design}.d)  was maintained  for five  hours, after  which the
`hot'  measurement proceeded  in exactly  the same  way as  the `cold'
measurement described  in the previous paragraph.  In  addition to the
sensitivity  calibrations,  we  also  performed  a  set  of  abundance
calibrations (`slow cals') by  expanding a known volume of calibration
gas into the  extraction line, including the target  volume.  This was
done at both room temperature and 700 $^{\circ}$C, using exactly
the same analytical procedure as for a real target measurement.

\section{Data reduction}\label{sec:datareduction}

The raw measurements  were in units of Amp\`{e}res  ($^4$He) and Hertz
(all  other  nuclides).  They   were  converted  to  atomic  units  by
sensitivity and abundance calibrations (the aforementioned `fast cals'
and `slow  cals', respectively).  The resulting  values were corrected
for   non-atmospheric   blanks,   shielding,  solar   modulation   and
tritiogenic  helium, following procedures  described in  the following
paragraphs.

\subsection{Blank corrections}\label{sec:blank}

Pressure inside the targets was  1-2 $\times$ 10$^{-2}$ mbar after one
year of exposure.  While this low pressure showcases the effectiveness
of  the  pre-exposure degassing  and  the  double  seal, the  inferred
amounts of non-cosmogenic $^3$He in the blank are far from trivial and
must  be  corrected  for.   The  easiest  solution  is  to  assume  an
atmospheric    isotopic    composition     (column    9    of    Table
\ref{tab:3HeData}).   However, the  $^3$He/$^4$He-ratio  for the  cold
step  of  blank  target   T13  was  17.7  $\pm$  1.9  $\times10^{-6}$,
significantly  higher than  the atmospheric  value (1.399  $\pm$ 0.013
$\times10^{-6}$, \citet{porcelli2002}).  The  hot steps were closer to
atmosphere: 3.3 $\pm$  1.1 $\times10^{-6}$ for T13, and  4.3 $\pm$ 1.5
$\times10^{-6}$  for a  second blank  target, T16  (column 7  of Table
\ref{tab:3HeData}).  The  non-atmospheric helium compositions  imply a
fractionated blank, possibly due to diffusion of helium and/or tritium
\citep{tilles1962} through the valves and container wall.  We used the
measured isotopic ratios of the blank targets for the blank correction
of all our targets (column 11 of Table \ref{tab:3HeData}), and applied
the  same procedure  for the  neon  measurements (column  15 of  Table
\ref{tab:21NeData}).  The  blank corrections  have a strong  effect on
the low  temperature helium  measurements of the  low-altitude targets
(Z\"{u}rich,   Davos,   S\"{a}ntis);   a   moderate  effect   on   the
high-altitude targets  (Jungfraujoch, Monte Rosa); and  a minor effect
on  the high temperature  measurements, including  neon.  Fortunately,
most  of  the  helium and  all  of  the  neon  were released  at  high
temperature (Section \ref{sec:results}).

\subsection{Shielding}

No topographic shielding correction  was required because all the high
elevation targets were installed on  the summits of mountains, and the
low  elevation targets (Z\"{u}rich  and Davos)  were located  in broad
valleys.   Snow cover  was either  removed  or negligible.   A 1.5  \%
correction was associated with shielding by the container wall (2.6 mm
steel) and an  additional 1.5 \% with the roofs at  the Monte Rosa and
Jungfraujoch  sites  (2 mm  aluminium  plus  2  cm wood).   A  nominal
uncertainty of 50\% was assigned to these corrections.

\subsection{Solar modulation}

The experiment took place during  a solar minimum, causing higher than
normal production rates.  One of our targets was installed adjacent to
the IGY neutron monitor at the Jungfraujoch neutron observatory, which
has  been in  continuous operation  since  1958.  We  have scaled  our
results to the average neutron signal over the past four solar cycles.
This correction  lowered our longer term production  rate estimates by
4.6\% (Figure \ref{fig:sun}).

\subsection{Tritiogenic helium}\label{sec:tritium}

The cosmogenic  $^3$He measured in natural samples  is either directly
derived from  spallogenic reactions  on heavier nuclides,  or it  is a
secondary radiogenic product of  spallogenic tritium. The half life of
$^3$H is 12.32  years, which is short compared to  the time periods of
interest in exposure dating, but  long in comparison with the duration
of our  artificial target  experiment. Therefore, the  apparent $^3$He
production rates measured by  our artificial targets are significantly
lower  than  the effective  $^3$He  production  rate  relevant to  TCN
studies. The relative amount of $^3$He  measured at a time t after the
retrieval of  a target that  has been exposed  for a time period  T is
given by:

$$
\frac{[^3He]_t}{[^3He]_{\infty}}  = 
1 - \frac{e^{-\lambda t}\left(1 - e^{-\lambda T}\right)}{\lambda T \left(1 +
    B\right)}
$$

where  B is  the $^3$He/$^3$H-branching  ratio, which  is  unknown for
spallation  by neutrons,  but  has been  estimated  by simulation  and
proton-based  accelerator  experiments  to  be $\sim$0.9  for  Si  and
$\sim$1.06 for  O \citep{leya2009}.  Therefore,  we will assume B  = 1
for quartz in the following.  \\

In the  case of  water targets, it  is straightforward to  measure the
branching  ratio. Because  tritium binds  to  oxygen in  water, it  is
retained  in the target  during $^3$He  measurement. By  remeasuring a
target some time after release  of the `primary' $^3$He, the amount of
tritiogenic helium can be determined, and a branching ratio calculated
\citep{brown2000}.   Things are somewhat  more complicated  for quartz
targets, because some if not all  of the tritium may escape during the
heating step.  Therefore, a  second target is  needed to  estimate the
$^3$He/$^3$H-branching ratio  in quartz.  Thus,  our duplicate targets
serve a  dual purpose. In addition  to allowing a  backup solution for
failed measurements,  they also provide  the only way to  avoid making
assumptions  about the  branching ratio.  We intend  to  analyse those
duplicate targets that have not yet been measured, especially the high
altitude targets from  Monte Rosa and Jungfraujoch, five  to ten years
from  now.   In the  meanwhile,  it is  still  necessary  to assume  a
branching ratio for the tritium-correction.

\section{Results}\label{sec:results}

Targets  T13, T10,  T12, T8  and T9  were measured  in  November 2007.
Target  T8 (S\"{a}ntis)  suffered  from an  atmospheric leak,  causing
anomalously high -- but still  measurable -- pressures and raw signals
significantly  higher  than those  of  the  Monte  Rosa target  (Table
\ref{tab:3HeData}).  The leak  precluded the $^{21}$Ne measurement for
T8,  but after  the  blank  correction, the  $^3$He  signal was  still
meaningful albeit  imprecise. In December 2007,  following an accident
unrelated to our experiment, the extraction line was contaminated with
water,  and  the mass  spectrometer  was out  of  service  for half  a
year. When the  duplicate S\"{a}ntis target (T7) was  measured in July
of  2008,  sensitivity  of  the   machine  had  dropped  by  a  third.
Sensitivity increased  again by about  10\% in September of  2008, but
this  was associated  with a  marked increase  in the  residual blank,
which   compromised   the  lowest   elevation   measurement  (T15   --
Z\"{u}rich).  After  a thorough servicing  of the extraction  line and
mass  spectrometer, blanks  were finally  reduced and  the sensitivity
restored to  the original levels.  A  duplicate Z\"{u}rich measurement
(T2)  was made  as  well as  an  additional blank  target (T16).   The
measurements  of the  `failed' target  measurements (T8  and  T15) are
given  in Table \ref{tab:3HeData}  for the  sake of  completeness, but
will not be considered in the  remainder of this paper, which will use
the duplicate target measurements  (T7 and T2) instead.  Therefore, in
the following  paragraphs, `S\"{a}ntis' is equivalent  to `target T7',
and `Z\"{u}rich' is equivalent to `target T2'.  \\

The  helium release spectra  indicate that  between 1  and 9\%  of the
cosmogenic   $^3$He   was   measured   in   the   cold   step   (Table
\ref{tab:3HeData}).  This confirms earlier experiments indicating that
$^3$He   quickly  diffuses   out   of  quartz   at  room   temperature
\citep{shuster2005}.  The  remaining $^3$He was  released at 700
$^{\circ}$C.    The   $^3$He/$^4$He   isotopic  ratio   systematically
increases with elevation,  which is a clear sign  of cosmogenic helium
production.  Non-atmospheric blank corrections  were 5\% for the Monte
Rosa target, 10\% for Jungfraujoch, 30\% for S\"{a}ntis and Davos, and
50\%  for  Z\"{u}rich  (Table  \ref{tab:3HeData}).  During  the  delay
caused by the aforementioned  technical problems, spallogenic $^3$H in
the targets decayed by about 5\%, resulting in an equivalent change in
the tritiogenic helium correction for the targets measured in 2008 (T7
and later), compared to the targets measured in 2007 (T8 and earlier).
\\

Virtually no $^{21}$Ne was measured  in the cold step, with nearly all
the   cosmogenic  $^{21}$Ne   being  released   at   700  $^{\circ}$C.
Re-extraction measurement at 800 $^{\circ}$C for the Monte Rosa target
showed that the small amount  of gas that remained at that temperature
had  an atmospheric  composition (Table  \ref{tab:21NeData}).   On the
neon  three-isotope plot, all  targets are  within 2$\sigma$  from the
spallation  line (Figure \ref{fig:3-isotopes}).   The cold  neon steps
(not  shown) and  blank  targets  (T13 and  T16)  have an  atmospheric
composition.  Precision of the Z\"{u}rich target (T2) was insufficient
to distinguish it from atmosphere, reflecting the limits of our target
design. Both  in the blanks  and in the  800$^{\circ}$C re-extraction,
the neon composition is slightly fractionated. This is consistent with
the  (much   stronger)  fractionation  observed   for  helium.   Blank
corrections for  neon were less than  5\% except for  the Davos target
(8\%).  This  is  much  smaller  than the  helium  blank  corrections.
Furthermore, blank  corrections are  inherently `safer' for  neon than
helium  because  the  former  has  three  isotopes,  facilitating  the
detection of non-atmospheric components (Figure \ref{fig:3-isotopes}).
So whereas neon is more challenging  to measure than helium due to the
lower  production rates  and  higher degassing  temperature, the  data
reduction is easier because blanks are  less of an issue, and there is
no significant radiogenic source.  \\

Although an empty  target container (exposed in Tibet  since 2005) was
not retrieved for  logistical reasons, it is possible  to evaluate the
production  of cosmogenic  noble  gases from  the  container walls  by
considering the $^3$He/$^{21}$Ne production rate ratio.  We may safely
assume  that  $^3$He would  be  affected  more  than $^{21}$Ne,  first
because  steel/quartz production rate  ratios are  considerably higher
for $^3$He than for $^{21}$Ne and second because helium should diffuse
more efficiently from the walls  into the container than neon.  Hence,
the $^3$He/$^{21}$Ne-ratio in the  empty targets should be higher than
the quartz value.  The  weighted mean of our measured $^3$He/$^{21}$Ne
ratios is 7.0  $\pm$ 0.9 (2$\sigma$) whereas in  natural quartz it may
be about 6.7 \citep{masarik1995}.  Thus, we believe that although some
of the  measured $^{3}$He and  $^{21}$Ne may come from  the container,
this  fraction is  small.   At any  rate,  it is  unimportant for  the
calculation  of  a  production   rate  attenuation  length,  which  is
discussed next.

\section{Attenuation lengths}\label{sec:attenuation}

The main goal of the artificial target experiment was to determine the
production rate  attenuation length.  Because  all our targets  had an
identical  design and  were  exposed under  identical conditions,  all
systematic  errors  should  cancel   out  in  the  calculation  of  an
attenuation  length.   After  scaling  the measurements  to  a  common
reference  latitude (Davos)  using  the latitudinal  scaling model  of
\citet{desilets2006a}, our best estimates are 132.7 $\pm$ 5.7 g/cm$^2$
for the  $^3$He attenuation length and  135 $\pm$ 25  g/cm$^2$ for the
$^{21}$Ne  attenuation  length  (Figure  \ref{fig:attenuation}).   The
MSWDs of  2.3 and 0.32  indicate slight over- and  underdispersion for
$^3$He and $^{21}$Ne, respectively \citep{mcintyre1966}.  Overall, the
fit is very good and the precision of the attenuation lengths (4\% for
$^3$He)  is comparable  to or  better  than that  obtained by  natural
calibration experiments \citep[e.g.,  ][]{desilets2006b}.  In order to
compare our attenuation length estimates with existing scaling models,
the  reference  cutoff  rigidity  of  4.36  GV  was  converted  to  an
equivalent  dipolar   geomagnetic  latitude  of   43.47$^{\circ}$  and
geomagnetic inclination  of 62.19$^{\circ}$.  Using  these latitudinal
parameters, scaling factors were calculated for the atmospheric depths
of our  target locations according to various  scaling procedures, and
for each  scaling procedure a  single exponential curve was  fitted to
the  synthetic  data,  exactly  as  was done  for  the  actual  target
measurements.   The resulting  attenuation lengths  compare favourably
with  our   estimates,  ranging  from   146  g/cm$^2$  \citep{lal1991,
  stone2000} to  131 g/cm$^2$ \citep{dunai2000,  desilets2003} and 133
g/cm$^2$  \citep{desilets2006a}.   All  calculations were  done  using
CosmoCalc 1.4 \citep{vermeesch2007c}.

\section{Production rates}\label{sec:P}

Determining  cosmogenic  noble gas  production  rates from  artificial
quartz targets is challenging because of the substantial uncertainties
associated      with     the     various      corrections     (Section
\ref{sec:datareduction}).   This  is   especially  the  case  for  the
$^3$He/$^3$H  branching  ratio which,  as  explained  before, we  have
assumed to  be one (Section \ref{sec:tritium}).   Given these caveats,
the production rate estimates  are remarkably consistent with previous
determinations.
\\

The corrected measurements were scaled  to sea level and high latitude
(SLHL) using  the empirical $^3$He  attenuation length of  132.7 $\pm$
5.7 g/cm$^2$ and cutoff rigidity  values provided by the University of
Bern ({\tt  http://cos\-ray.uni\-be.ch}).  The five  $^3$He production
rate estimates  scatter between 105  and 131 at/g/yr,  with analytical
uncertainties between  11\% (S\"{a}ntis) and  21\% (Z\"{u}rich)(Figure
\ref{fig:productionrates}).   The  weighted  mean production  rate  of
110.5 $\pm$ 6.7 at/g/yr is  in good agreement with physics-based model
calculations  \citep{masarik1995}.   The   observed  scatter  is  well
explained by the analytical uncertainty alone, as
indicated by the MSWD of 0.71.  \\

The $^{21}$Ne production  rate estimates range from 15  to 19 at/g/yr,
with analytical  uncertainties of 22  to 80\%, which  is substantially
larger than the helium  production rate uncertainties.  The relatively
low  MSWD (0.20)  indicates underdispersion  and, therefore,  the neon
uncertainties are probably  somewhat overestimated.  The weighted mean
of the five production rate estimates is 15.4 $\pm$ 2.1 at/g/yr, which
is   also  close   to  the   accepted   values  \citep{niedermann1994,
  masarik1995,  niedermann2000, goethals2009,  amidon2009}, especially
considering that  solar activity has been consideraby  higher over the
past  five   decades  than  during   the  11,000  years   before  that
\citep{solanki2004},  causing  lower  than  normal  instantaneous  TCN
production rates.

\section{Conclusions and outlook}\label{sec:conclusions}

We  successfully measured  cosmogenic $^3$He  and $^{21}$Ne  in quartz
after one  year of  exposure in the  Swiss Alps. After  correcting for
non-atmospheric blanks,  shielding, tritiogenic helium  production and
solar modulation, the cosmogenic noble gas production rates agree well
with previous determinations. Production  rates, however, were not our
primary concern  for two reasons.  First,  it is hard to  rule out all
systematic  errors  associated with  the  aforementioned corrections.  
Second, production rates vary greatly  over time in reponse to changes
in Earth's magnetic field and  variations in solar activity, making it
difficult to  compare short- and long-term production  rates with each
other.   Nevertheless,  the  good  agreement of  our  production  rate
estimates   with  previous   determinations,   and  particularly   the
$^3$He/$^{21}$Ne production  rate ratio, gives us  great confidence in
the robustness of our method.   The main strength of artificial target
experiments  is  in  measuring  attenuation  lengths,  again  for  two
reasons.  First,  all systematic errors cancel out  in the calculation
of an  attenuation length.  Second, our  short-term attenuation length
estimates  can be  directly compared  with short-term  neutron monitor
surveys.  The  excellent agreement between  the former and  the latter
indicates that neutron monitors are indeed a good basis for production
rate scaling  models.  And in  contrast to what others  have suggested
\citep{amidon2008}, altitudinal scaling of noble gases also appears to
be identical to  that of other cosmogenic nuclides,  such as $^{36}$Cl
or  $^{10}$Be.  These simplifying  results are  a nice  departure from
many other  recent developments  in cosmogenic nuclide  science, which
has rapidly grown increasingly complex over the past few years
\citep[e.g., ][]{pigati2004, staiger2007, lifton2008}.  \\

This  is an ongoing  experiment.  As  discussed before,  the duplicate
targets  of Monte  Rosa, Jungfraujoch,  and  Davos have  not yet  been
measured and are kept shielded  from cosmic rays. Measuring the $^3$He
content of  these targets in  a few years  time will provide  a direct
estimate of  the $^3$He/$^3$H branching  ratio in quartz. As  an added
bonus, because  the temperature in  the storage room is  constant, the
amount of $^3$He  measured in the cold step  relative to that measured
in the hot step will put  further constraints on the ease of diffusion
of   cosmogenic   helium   out   of   quartz   at   room   temperature
\citep{shuster2005}.  During the summer  of 2007, the targets reported
in this paper were replaced by  a new set of identical targets.  These
replacement targets have two purposes. First, they will be exposed for
longer than one year, which  will further improve the precision of our
experiment. Second, subsequent replacements will track an entire solar
cycle by sequential artificial target measurements over the next
decade.  \\

The precision  of our attenuation  length measurements rivals  that of
natural calibration samples, and the  accuracy is likely to be better,
as our artificial targets circumvent the geological problems (erosion,
snow, and  vegetation cover) affecting  natural targets.  Furthermore,
our artificial  targets can  be used anywhere  on the  planet, whereas
suitable  lithologies and exposure  histories for  natural calibration
experiments  are  rare. Nevertheless,  natural  calibration sites  are
still  useful as  they  integrate cosmogenic  nuclide production  over
millennial time scales that are  more relevant to exposure dating than
the  one  year  time  scale  of our  experiment.   Thus,  natural  and
artificial targets  have complementary strengths  which, together, can
be used to further test and improve scaling models.

\section*{Acknowledgments}

Swift  yet detailed  reviews by  William Amidon  (Caltech)  and Samuel
Niedermann (GFZ  Potsdam) are gratefully acknowledged.   PV would also
like to  thank Rolf B\"{u}tikofer  (University of Bern)  for providing
the neutron monitor measurements (Figure \ref{fig:sun}), Louise Wilson
(University  of  Bern) for  the  permits  to  access the  Jungfraujoch
neutron observatory,  Arthur Kunz (MeteoSwiss) for  granting access to
meteorological stations at all  target locations below Monte Rosa, and
Giorgio Tiraboschi  (Club Alpino Italiano) for granting  access to the
Cabana Margarita  on Monte Rosa.  This work  was financially supported
by  Swiss Nationalfonds grant  No. 200020-105220/1  and a  Marie Curie
Fellowship  of  the European  Union  (CRONUS-EU  network, RTN  project
reference 511927).

\begin{thebibliography}{34}
\expandafter\ifx\csname natexlab\endcsname\relax\def\natexlab#1{#1}\fi
\expandafter\ifx\csname url\endcsname\relax
  \def\url#1{\texttt{#1}}\fi
\expandafter\ifx\csname urlprefix\endcsname\relax\def\urlprefix{URL }\fi

\bibitem[{Amidon et~al.(2008)Amidon, Farley, Burbank, and
  Pratt-Sitaula}]{amidon2008}
Amidon, W., Farley, K., Burbank, D., Pratt-Sitaula, B., 2008. Anomalous
  cosmogenic $^3${H}e production and elevation scaling in the high {H}imalaya.
  Earth and Planetary Science Letters 265~(1-2), 287--301.

\bibitem[{Amidon et~al.(2009)Amidon, Rood, and Farley}]{amidon2009}
Amidon, W., Rood, D., Farley, K., 2009. Cosmogenic $^3${H}e and $^{21}${N}e
  production rates calibrated against $^{10}${B}e in minerals from the {C}oso
  volcanic field. Earth and Planetary Science Letters (in press).

\bibitem[{Baur(1980)}]{baur1980}
Baur, H., 1980. Numerische simulation und praktische erprobung einer
  rotationssymmetrischen ionenquelle f\"{u}r gasmassenspektrometer. Ph.D.
  Thesis, ETH-Z\"{u}rich No. 6596.

\bibitem[{Baur(1999)}]{baur1999}
Baur, H., 1999. A noble-gas mass spectrometer compressor source with two orders
  of magnitude improvement in sensitivity. EOS, Transactions of the American
  Geophysical Union 80, F1118.

\bibitem[{Brown et~al.(2000)Brown, Trull, Jean-Baptiste, Raisbeck, Bourl\`{e}s,
  Yiou, and Marty}]{brown2000}
Brown, E.~T., Trull, T.~W., Jean-Baptiste, P., Raisbeck, G., Bourl\`{e}s, D.,
  Yiou, F., Marty, B., 2000. Determination of cosmogenic production rates of
  $^{10}${B}e, $^{3}${H}e and $^3${H} in water. Nuclear Instruments and Methods
  in Physics Research Section B 172, 873--883.

\bibitem[{Desilets and Zreda(2003)}]{desilets2003}
Desilets, D., Zreda, M., 2003. Spatial and temporal distribution of secondary
  cosmic-ray nucleon intensities and applications to in situ cosmogenic dating.
  Earth and Planetary Science Letters.

\bibitem[{Desilets and Zreda(2006)}]{desilets2006b}
Desilets, D., Zreda, M., 2006. Elevation dependence of cosmogenic $^{36}${C}l
  production in {H}awaiian lava flows. Earth and Planetary Science Letters 246,
  277--287.

\bibitem[{{Desilets} et~al.(2006){Desilets}, {Zreda}, and
  {Prabu}}]{desilets2006a}
{Desilets}, D., {Zreda}, M., {Prabu}, T., 2006. {Extended scaling factors for
  in situ cosmogenic nuclides: New measurements at low latitude}. Earth and
  Planetary Science Letters 246, 265--276.

\bibitem[{Dunai(2000)}]{dunai2000}
Dunai, T., 2000. Scaling factors for production rates of in situ produced
  cosmogenic nuclides: a critical reevaluation. Earth and Planetary Science
  Letters 176, 157--169.

\bibitem[{Dunai(2001)}]{dunai2001}
Dunai, T., 2001. Influence of secular variation of the geomagnetic field on
  production rates of in situ produced cosmogenic nuclides. Earth and Planetary
  Science Letters 193, 197--212.

\bibitem[{Goethals et~al.(2009)Goethals, Hetzel, Niedermann, Wittmann, Fenton,
  Christl, Kubik, and von Blanckenburg~F.}]{goethals2009}
Goethals, M.~M., Hetzel, R., Niedermann, S., Wittmann, H., Fenton, C.~R.,
  Christl, M., Kubik, P.~W., von Blanckenburg~F., 2009. An accurate
  experimental determination of cosmogenic $^{10}${B}e/$^{21}${N}e and
  $^{26}${A}l/$^{21}${N}e production ratios in quartz. Earth and Planetary
  Science Letters, (in review).

\bibitem[{Graf et~al.(1996)Graf, Marti, and Wiens}]{graf1996}
Graf, T., Marti, K., Wiens, R., 1996. The $^{21}${N}e production rate in a {S}i
  target at mountain altitudes. Radiocarbon 38, 155.

\bibitem[{Graham et~al.(2000)Graham, Barry, Ditchburn, and
  Whitehead}]{graham2000}
Graham, I.~J., Barry, B.~J., Ditchburn, R.~G., Whitehead, N.~E., 2000.
  Validation of cosmogenic nuclide production rate scaling factors through
  direct measurement. Nuclear Instruments and Methods in Physics Research
  Section B 172, 802--805.

\bibitem[{Kober(2004)}]{kober2004}
Kober, F., 2004. Quantitative analysis of the topographic evolution of the
  {A}ndes of northern {C}hile using cosmogenic nuclides. PhD thesis,
  ETH-Z\"{u}rich No. 15858.

\bibitem[{Lal(1991)}]{lal1991}
Lal, D., 1991. Cosmic ray labelling of erosion surfaces: {\it in situ} nuclide
  production rates and erosion models. Earth and Planetary Science Letters 104,
  424--439.

\bibitem[{Lal et~al.(1960)Lal, Arnold, and Honda}]{lal1960}
Lal, D., Arnold, J., Honda, M., 1960. Cosmic-ray production rates of {B}e$^7$
  in oxygen, and {P}$^{32}$, {P}$^{33}$, {S}$^{35}$ in argon at mountain
  altitudes. Physical Review 118~(6), 1626--1632.

\bibitem[{Leya and Masarik(2009)}]{leya2009}
Leya, I., Masarik, J., 2009. Cosmogenic nuclides in stony meteorites revisited.
  Meteoritics and Planetary Science, (in revision).

\bibitem[{Lifton et~al.(2008)Lifton, Smart, and Shea}]{lifton2008}
Lifton, N., Smart, D.~F., Shea, M.~A., 2008. Scaling time-integrated in situ
  cosmogenic nuclide production rates using a continuous magnetic model. Earth
  and Planetary Science Letters 268, 190--201.

\bibitem[{{Masarik} and {Reedy}(1995)}]{masarik1995}
{Masarik}, J., {Reedy}, R.~C., 1995. {Terrestrial cosmogenic-nuclide production
  systematics calculated from numerical simulations}. Earth and Planetary
  Science Letters 136, 381--395.

\bibitem[{{McIntyre} et~al.(1966){McIntyre}, {Brooks}, {Compston}, and
  {Turek}}]{mcintyre1966}
{McIntyre}, G.~A., {Brooks}, C., {Compston}, W., {Turek}, A., 1966. {The
  Statistical Assessment of Rb-Sr Isochrons}. Journal of Geophysical Research
  71, 5459--5468.

\bibitem[{{Niedermann}(2000)}]{niedermann2000}
{Niedermann}, S., 2000. {The $^{21}${N}e production rate in quartz revisited}.
  Earth and Planetary Science Letters 183, 361--364.

\bibitem[{Niedermann(2002)}]{niedermann2002}
Niedermann, S., 2002. {Cosmic-Ray-Produced Noble Gases in Terrestrial Rocks:
  Dating Tools for Surface Processes}. Vol.~47 of Reviews in Mineralogy and
  Geochemistry.

\bibitem[{Niedermann et~al.(1994)Niedermann, Graf, Kim, Kohl, Marti, and
  Nishiizumi}]{niedermann1994}
Niedermann, S., Graf, T., Kim, J., Kohl, C., Marti, K., Nishiizumi, K., 1994.
  Cosmic-ray-produced $^{21}${N}e in terrestrial quartz: the neon inventory of
  {S}ierra {N}evada quartz separates. Earth and Planetary Science Letters
  125~(1-4), 341--355.

\bibitem[{Nishiizumi et~al.(1996)Nishiizumi, Finkel, Klein, and
  Kohl}]{nishiizumi1996}
Nishiizumi, K., Finkel, R.~C., Klein, J., Kohl, C., 1996. Cosmogenic production
  of $^7${B}e and $^{10}${B}e in water targets. Journal of Geophysical Research
  101, 22,225--22,232.

\bibitem[{Pigati and Lifton(2004)}]{pigati2004}
Pigati, J., Lifton, N., 2004. Geomagnetic effects on time-integrated cosmogenic
  nuclide production with emphasis on in situ $^{14}${C} and $^{10}${B}e. Earth
  and Planetary Science Letters 226, 193--205.

\bibitem[{Porcelli et~al.(2002)Porcelli, Ballentine, and Wieler}]{porcelli2002}
Porcelli, D., Ballentine, C.~J., Wieler, R., 2002. An overview of noble gas
  geochemistry and cosmochemistry. Vol.~47 of Reviews in Mineralogy and
  Geochemistry.

\bibitem[{Sch\"{a}fer(2000)}]{schaefer2000}
Sch\"{a}fer, J.~M., 2000. Reconstruction of landscape evolution and continental
  paleoglaciations using in situ cosmogenic nuclides. Ph.D. Thesis,
  ETH-Z\"{u}rich No. 13542.

\bibitem[{Shuster and Farley(2005)}]{shuster2005}
Shuster, D., Farley, K., 2005. Diffusion kinetics of proton-induced
  $^{21}${N}e, $^{3}${H}e, and $^{4}${H}e in quartz. Geochimica et Cosmochimica
  Acta 69, 2349--2359.

\bibitem[{{Solanki} et~al.(2004){Solanki}, {Usoskin}, {Kromer},
  {Sch{\"u}ssler}, and {Beer}}]{solanki2004}
{Solanki}, S.~K., {Usoskin}, I.~G., {Kromer}, B., {Sch{\"u}ssler}, M., {Beer},
  J., 2004. {Unusual activity of the Sun during recent decades compared to the
  previous 11,000 years}. Nature 431, 1084--1087.

\bibitem[{{Staiger} et~al.(2007){Staiger}, {Gosse}, {Toracinta}, {Oglesby},
  {Fastook}, and {Johnson}}]{staiger2007}
{Staiger}, J., {Gosse}, J., {Toracinta}, R., {Oglesby}, B., {Fastook}, J.,
  {Johnson}, J.~V., 2007. {Atmospheric scaling of cosmogenic nuclide
  production: Climate effect}. Journal of Geophysical Research (Solid Earth)
  112, B2205.

\bibitem[{Stone(2000)}]{stone2000}
Stone, J.~O., 2000. Air pressure and cosmogenic isotope production. Journal of
  Geophysical Research 105, 23,753--23,759.

\bibitem[{Strasky(2008)}]{strasky2008}
Strasky, S., 2008. Glacial response to global climate changes: cosmogenic
  nuclide chronologies from high and low latitudes. Ph.D. Thesis,
  ETH-Z\"{u}rich No. 17569.

\bibitem[{Tilles(1962)}]{tilles1962}
Tilles, D., 1962. {A Room-Temperature Diffusion Constant for Hydrogen in
  Proton-irradiated Steel}. Nature 194~(4835), 1273--1274.

\bibitem[{{Vermeesch}(2007)}]{vermeesch2007c}
{Vermeesch}, P., 2007. {CosmoCalc: An Excel add-in for cosmogenic nuclide
  calculations}. Geochemistry, Geophysics, Geosystems 8, 8003.

\end{thebibliography}


%\bibliography{d:/papers/biblio}
%\bibliography{biblio}
%\bibliographystyle{plainnat}
%\bibliographystyle{elsarticle-harv}

\clearpage

\listoffigures

\listoftables

\clearpage

\section*{Figures}

\begin{figure}[h]
  \centering
  \includegraphics[width=.85\textwidth]{design.pdf}
  \caption{
a -- Target material: artificially grown single 
    quartz  crystal before  crushing.  b  -- Custom-built  furnace for
    quartz degassing.   Note that the pinch-off clamp  (missing in the
    picture) is  mounted and closed before decoupling  the target from
    the  system.    c  --  Prototype  container   outfitted  with  two
    thermocouples  to verify  thermal  equilibration. d  -- After  two
    hours of  heating at an  external temperature of  900 $^{\circ}$C,
    the internal parts of the  quartz in the prototype container reach
    a temperature  $\sim$850 $^{\circ}$C, ensuring  complete degassing
    of both neon and helium.}
  \label{fig:design}
\end{figure}

\clearpage
\begin{figure}[htbp]
  \centering
  \includegraphics[width=.8\textwidth]{sun.pdf}
  \caption{Sunspot  number (grey)  and IGY  neutron monitor  counts at
    Jungfraujoch (black) over the  past 50 years.  The vertical dashed
    lines mark the exposure time of our targets, the horizontal dotted
    lines the average number of counts over the past five solar cycles
    (bottom) and the duration of the experiment (top).}
  \label{fig:sun}
\end{figure}

\clearpage
\begin{figure}[htbp]
  \centering
  \includegraphics[width=.8\textwidth]{3-isotopes.pdf}
  \caption{Neon three-isotope plot (700 $^{\circ}$C steps). 
The star marks the atmospheric composition,
    the spallation line is shown in grey \citep{niedermann2002}. Error
    symbols are 2$\sigma$.}
  \label{fig:3-isotopes}
\end{figure}

\clearpage
\begin{figure}[htbp]
  \centering
  \includegraphics[width=\textwidth]{attenuation.pdf}
  \caption{Production rate attenuation lengths for $^3$He (right) and $^{21}$Ne (left).
    Error  bars  are  2$\sigma$.  Targets  are, from  top  to  bottom:
    Z\"{u}rich (T2), Davos  (T9), S\"{a}ntis (T7), Jungfraujoch (T10),
    and Monte Rosa (T12).}
  \label{fig:attenuation}
\end{figure}

\clearpage
\begin{figure}[htbp]
  \centering
  \includegraphics[width=.5\textwidth]{productionrates.pdf}
  \caption{Solar modulation corrected production rates at sea level and high 
latitude for $^3$He (top) and $^{21}$Ne (bottom). Error bars are 2$\sigma$, the grey
lines represent the weighted means.}
  \label{fig:productionrates}
\end{figure}

\clearpage

\section*{Tables}

\begin{table}[htbp]
  \centering
% Table generated by Excel2LaTeX from sheet 'Tables'
\begin{tabular}{cccccccccccc}
  location &     T &   $^3$He &   2$\sigma$ &   $^4$He &   2$\sigma$ &    3/4 &   2$\sigma$ &  $^3$He$^{\dagger}$ &   2$\sigma$ & $^3$He$^{\ddagger}$ &   2$\sigma$ \\
           &     [$^{\circ}$C] &    [$\times$10$^3$] &     &     [$\times$10$^7$] &    &     [$\times$10$^{-6}$] &     &    [$\times$10$^3$] &    &    [$\times$10$^3$] &    \\
\hline
\hline
blank &    20 &    66.3 &     6.6 &  374 &   11 &  17.7 &   1.9 &  61.1 &   6.6 &     0 &   10 \\
(T13)           &   700 &    14.6 &     4.9 &  442 &   13 &   3.3 &   1.1 &   8.4 &   4.9 &     0 &   7.0 \\
\hline
Jungfrau &    20 &   149 &    13 &  478 &   14 &  31.3 &   2.8 & 143 &  13 &  65 &  16 \\
(T10)         &   700 &   745 &    41 &  430 &   13 & 173 &  11 & 740 &  40 & 731 &  41 \\
\hline
Monte Rosa &    20 &   160 &    11 &  377 &   11 &  42.5 &   3.2 & 155 &  11 &  93 &  13 \\
(T12)       &   700 &  1488 &    70 &  593 &   17 & 251 &  14 &     1479 &  70 &     1468 &  71 \\
           &   800 &     9.1 &     4.0 &  438 &   13 &   2.1 &   0.9 &   2.9 &   4.0 &   2.9 &   4.0 \\
\hline
S\"{a}ntis &    20 &   174 &    17 &  806 &   23 &  21.6 &   2.2 & 163 &  17 &  31 &  23 \\
(T8$^*$)           &   700 &   453 &    31 & 2804 &   81 &  16.1 &   1.2 & 413 &  31 & 360 &  44 \\
\hline
Davos &    20 &    81.2 &     8.4 &  424 &   12 &  19.2 &   2.1 &  75.3 &   8.4 &   6 &  12 \\
(T9)         &   700 &   206 &    16 &  461 &   14 &  44.7 &   3.8 & 199 &  16 & 191 &  17 \\
\hline
S\"{a}ntis &    20 &    83 &    12 &  444 &   17 &  18.7 &   2.9 &  77 &  12 &   4 &  15 \\
(T7)        &   700 &   431 &    29 &  406 &   17 & 106.2 &   8.4 & 425 &  29 & 414 &  30 \\
\hline
Z\"{u}rich &    20 &   191 &    17 &  463 &   16 &  41.4 &   4.0 & 185 &  17 & 109 &  19 \\
(T15$^{**}$)         &   700 &   195 &    21 &  586 &   20 &  33.2 &   3.8 & 187 &  21 & 170 &  23 \\
\hline
Z\"{u}rich &    20 &   115.6 &     9.0 &  640 &   19 &  18.1 &   1.5 & 106.7 &   9.0 &   2 &  15 \\
(T2)          &   700 &   123 &    16 &  430 &   13 &  28.6 &   3.8 & 117 &  16 & 104 &  17 \\
\hline
blank (T16$^{***}$) &   700 &    18.6 &     6.6 &  430 &   13 &   4.3 &   1.5 &  12.6 &   6.6 &     0 &   9.3 \\
\end{tabular}
\caption{Helium measurements (atoms) in chronological order of measurement. 
$\dagger$ -- cosmogenic $^3$He assuming an atmospheric blank; 
$\ddagger$ -- cosmogenic $^3$He using the measured blank; * -- leaking target; 
** -- high residual blank in the mass spectrometer; *** -- cold step lost.}
\label{tab:3HeData}
\end{table}

\clearpage
\begin{table}[htbp]
  \centering
% Table generated by Excel2LaTeX from sheet 'Tables'
\begin{tabular}{cccccccccccccccc}
  location &     T &  $^{20}$Ne &   2$\sigma$ &  $^{21}$Ne &   2$\sigma$ &  $^{22}$Ne &   2$\sigma$ & 21/20 &   2$\sigma$ & 22/20 &   2$\sigma$ & $^{21}$Ne$\dagger$ &   2$\sigma$ &     $^{21}$Ne$\ddagger$ &   2$\sigma$ \\
           &   [$^{\circ}$C]    &     [$\times$10$^6$] &      &    [$\times$10$^3$] &     &     [$\times$10$^6$] &   &    [$\times$10$^{-3}$] &   &    [$\times$10$^{-3}$] &   &    [$\times$10$^3$] &   &    [$\times$10$^3$] &  \\
\hline
\hline
T13 &   700 & 603 &  18 &  1768 &    55 & 61.2 &  1.8 &  2.93 &  0.13 & 101.5 &   4.3 & -18 &  77 &     0 & 110 \\
\hline
T10 &   700 & 266.3 &   7.9 &   998 &    33 & 26.85 &  0.80 &  3.75 &  0.17 & 100.8 &   4.2 & 210 &  40 & 218 &  52 \\
\hline
T12 &   700 & 596 &  18 &  2184 &    68 & 61.4 &  1.8 &  3.66 &  0.16 & 103.0 &   4.3 & 420 &  86 & 440 & 110 \\
           &   800 & 306.0 &   9.0 &   907 &    29 & 30.93 &  0.92 &  2.97 &  0.13 & 101.1 &   4.2 &   2 &  40 &  11 &  55 \\
\hline
T9 &   700 & 176.9 &   5.2 &   585 &    19 & 18.16 &  0.53 &  3.31 &  0.14 & 102.6 &   4.3 &  61 &  24 &  66 &  33 \\
\hline
T7 &   700 &  81.4 &   3.3 &   354 &    17 &  8.33 &  0.35 &  4.35 &  0.27 & 102.4 &   6.0 & 113 &  19 & 112 &  23 \\
\hline
T15$^{**}$ &   700 & 142.7 &   5.1 &   436 &    20 & 13.72 &  0.51 &  3.06 &  0.18 &  96.2 &   4.9 &  14 &  25 &  11 &  34 \\
\hline
T2 &   700 &  53.3 &   3.5 &   182 &    13 &  5.58 &  0.36 &  3.41 &  0.32 & 104.6 &   9.6 &  24 &  16 &  23 &  18 \\
\hline
T16$^{***}$ &   700 &  46.3 &   1.4 &   137.8 &     6.1 &  4.66 &  0.16 &  2.98 &  0.16 & 100.6 &   4.6 &   0.8 &   7.4 &     0 &  11 \\
\end{tabular}
\caption{Neon measurements, in atoms. Annotations as in Table \ref{tab:3HeData}.}
\label{tab:21NeData}
\end{table}

\clearpage
\begin{table}[htbp]
  \centering
% Table generated by Excel2LaTeX from sheet 'Tables'
\begin{tabular}{cccccc}
  location & Monte Rosa (T12) & Jungfrau (T10) & S\"{a}ntis (T7) & Davos (T9) & Z\"{u}rich (T2) \\
\hline
latitude & 45.94        & 46.55 & 47.25 & 46.80 & 47.37\\
longitude & 7.87        & 7.98  & 9.34  & 9.84  & 8.54\\
 elevation &       4554 &       3571 &       2502 &       1560 &        560 \\
date closed & 02/08/2006 & 26/07/2006 & 25/04/2006 & 10/06/2006 & 02/08/2006 \\
   exposed & 25/08/2006 & 24/08/2006 & 05/09/2006 & 05/09/2006 & 06/09/2006 \\
 retrieved & 01/09/2007 & 27/08/2007 & 29/08/2007 & 29/08/2007 & 24/08/2007 \\
  measured & 26/11/2007 & 24/11/2007 & 07/07/2008 & 30/11/2007 & 05/11/2008 \\
   \# years &      1.018 &      1.008 &      0.980 &      0.980 &      0.964 \\
 shielding &     0.963 &     0.963 &     0.985 &     0.985 &     0.985 \\
\hline
 $^3$He [$\times$10$^3$] &    1561 &     796 &     418 &     197 &     107 \\
 2$\sigma$ &      72 &      44 &      34 &      21 &      23 \\
 $^3$He$_t$/$^3$He$_{\infty}$ &      0.507 &      0.507 &      0.524 &      0.507 &      0.533 \\
 P($^{3}$He) &       3140 &       1618 &        826 &        402 &        211 \\
 2$\sigma$ &        160 &         94 &         67 &         43 &         46 \\
 $^{21}$Ne [$\times$10$^3$] &     450 &     225 &     116 &      69 &      24 \\
 2$\sigma$ &     120 &      54 &      24 &      34 &      19 \\
 P($^{21}$Ne) &        450 &        225 &        116 &         69 &         24 \\
 2$\sigma$ &        120 &         54 &         24 &         34 &         19 \\
\hline
R$_c$ & 4.69    & 4.46  & 4.19  & 4.36  & 4.15\\
atm depth &        584 &        664 &        761 &        856 &        966 \\
S$_{\lambda}$   & 0.90 & 0.91 & 0.92 & 0.92 & 0.93\\
S$_{z}$ & 29.5  & 16.2  & 7.78  & 3.80  & 1.659\\
2$\sigma$ &     4.3   & 1.9 & 0.68 & 0.22 & 0.036\\
P($^3$He)$_{_{SLHL}}$   & 113 & 105 & 110 & 110 & 131\\
2$\sigma$ & 17 & 14   & 13   & 13  & 29\\
P($^{21}$Ne)$_{_{SLHL}}$        & 16.0  & 14.6  & 15.4  & 18.9  & 15\\
2$\sigma$ & 4.8 & 3.9   & 3.5   & 9.4   & 12
\end{tabular}
\caption{Data reduction for $^{3}$He and $^{21}$Ne.  P($\cdot$) =
  production rates in at/g/yr, R$_c$ = cutoff rigidity in GV,
  S$_{\lambda}$ = latitudinal scaling factor, S$_{z}$ = altitudinal
  scaling factor. Elevation in metres, atmospheric depth in g/cm$^2$,
  $^{3}$He and $^{21}$Ne in atoms.  The SLHL production rates (bottom)
  have been corrected for solar modulation, the raw production rates
  (middle) have not.}
  \label{tab:3He21Ne}
\end{table}

\end{document}
